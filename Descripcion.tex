\documentclass[a4paper,12pt]{/home/armando/Documentos/Cursos/LaTeX/Plantillas/lib/pub}
%\graphicspath{ {image/} }
\usepackage{wrapfig}
\usepackage[T1]{fontenc}
\usepackage{tgadventor}

\ptcSeccion{Blockchain Academy México}
\ptcTitulo{Stellar}
\ptcLogoSeccion{BAM}

\begin{document}
\putLogo
\protecoTitle

\renewcommand{\contentsname}{Índice General}
\tableofcontents
\newpage

\section{Descripción del curso}
	\subsection{Descripción general}
	Este curso te dará los conocimientos fundamentales para entender qué es blockchain, qué es el proyecto Stellar, por qué es tan útil y cuál es su propuesta de valor, también veremos los casos de uso, en la parte práctica veremos cómo hacer una cuenta para enviar y recibir pagos.
	\subsection{Objetivo}
		\subsubsection{Objetivo principal}
		Enseñar al alumno desde los conceptos teóricos del proyecto Stellar, sus casos de uso,las ventajas y desventajas de utilizar la plataforma; hasta como pueden crear wallets y realizar transacciones con Lumens.
		\subsubsection{Objetivos secundarios}
	\subsection{Perfil del egresado}
	El alumno será capaz de desarrollar una wallet y hacer transacciones con Lumnes, la criptomoneda de Stellar.
	\subsection{Requisitos de conocimientos}
	Es preferente haber tomado el curso de Blockchain101 sin embargo se dará una introducción a blockchain y daremos un repaso de antecedentes.
	\subsection{Requisitos de hardware}
	Computadora con acceso a internet y mínimo 4GB de RAM.
	\subsection{Requisitos de software}
	Sistema operativo Ubuntu o MacOS, si tienes Windows puedes instalar una máquina virtual.\\
	Python o NPM

\section{Temario}
	\subsection{Fundamentos de Blockchain}
	\begin{itemize}
		\item ¿Qué es blockchain?
		\item ¿Qué resuelve blockchain?
		\item Tecnologías que conforman blockchain
		\begin{itemize}
			\item Criptografía
			\item Red Peer to Peer
			\item Libro contable
			\item Protocolo de consenso
		\end{itemize}
		\item ¿Qué es una DLT?
		\item Tipos de blockchain
		\begin{itemize}
			\item Blockchain Público
			\item Blockchain Privado
			\item Blockchain Permisionado
			\item Blockchain Federados
		\end{itemize}
	\end{itemize}
	\subsection{Introducción a Stellar}
	\begin{itemize}
		\item ¿Qué es Stellar?
		\item ¿Qué es Ripple
		\item Historia de Stellar
		\item ¿Cómo funciona?
		\item Contratos inteligentes
		\item Ventajas de Stellar
		\item Casos de uso
		\begin{itemize}
			\item IBM Worldwire
			\item SatoshiPay
		\end{itemize}
		\item Stellar tokens
		\item Stellar assets
		\item Stellar wallets
	\end{itemize}
	\subsection{Stellar práctico}
	\begin{itemize}
		\item Crear una cuenta
		\item Enviar y recibir pagos
	\end{itemize}

\section{Descipción de cada módulo}
	\subsection{Fundamentos de Blockchain}
	El alumno se familiarizará con blockchain y las tecnologías que la conforman, conocerá los tipos de blockchain y qué es una DLT. Este módulo tiene como objetivo ser un repaso para los estudiantes que tomaron el curso de Blockchain 101 y una introducción para los estudiantes que no están familiarizados con la tecnología blockchain.\\
	\textbf{Tiempo estimado}: 3 horas.
	\subsection{Introducción a Stellar}
	Este módulo está enfocado a la parte teórica de Stellar. El alumno conocerá el proyecto Stellar así como su historia, sabrá cuáles son los casos de uso, las ventajas y desventajas. También hablaremos un poco del proyecto Ripple y la relación que tiene con Stellar.\\
	\textbf{Tiempo estimado}: 3 horas.
	\subsection{Stellar práctico}
	Este módulo está enfocado a la parte práctica de Stellar. El alumno será capaz de hacer transacciones y crear una cuenta en Stellar para enviar y recibir pagos.\\
	\textbf{Tiempo estimado}: 3 horas.

\section{Minuto a minuto}
\textbf{Lunes}
\begin{itemize}
	\item 8:00-8:15 Presentación del curso y de los instructores.
	\item 8:15-10:15 Introducción a blockchain
	\item 10:15-10:30 Descanso
	\item 10:30-11:30 Introducción a blockchain (Continuación)
	\item 11:30-12:00 Introducción a Stellar
\end{itemize}
\textbf{Martes}
\begin{itemize}
	\item 8:00-8:15 Resolver dudas de la clase anterior
	\item 8:15-8:30 Kahoot
	\item 8:30-10:15 Introducción a Stellar(Continuación)
	\item 10:15-10:30 Descanso
	\item 10:30-11:15 Introducción a Stellar(Continuación)
	\item 11:15-12:00 Stellar práctico
\end{itemize}
\textbf{Miércoles}
\begin{itemize}
	\item 8:00-8:15 Resolver dudas de la clase anterior
	\item 8:15-8:30 Kahoot
	\item 8:30-10:15 Stellar práctico (Continuación)
	\item 10:15-10:30 Descanso
	\item 10:30-11:00 Stellar práctico (Continuación)
	\item 11:00-12:00 Conclusión del curso y evaluaciones
\end{itemize}
\section{Guía de apoyo para cada módulo}
	\subsection{Fundamentos de Blockchain}
	\subsection{Introducción a Stellar}
	\subsection{Stellar práctico}
	
\section{Ejercicios y actividades de cada módulo}

\section{Kahoot de cada módulo}
	\subsection{Fundamentos de Blockchain}
	\subsection{Introducción a Stellar}
	\subsection{Stellar práctico}

\section{Evaluaciones}
	\subsection{Examen Final}
	\subsection{Encuesta al instructor}
	Se proponen 10 preguntas para evaluar al instructor o los instructores.
	\begin{enumerate}
		\item ¿Cómo calificarías la actitud del instructor?
		\item ¿Cómo calificarías los conocimientos del instructor?
		\item ¿De qué manera explico los temas el instructor?
		\item ¿El instructor faltó al curso?
		\item ¿El instructor llegó tarde al curso?
		\item ¿El instructor llevó material  para reforzar su explicación?
		\item ¿Cómo calificarías la claridad con la que el instructor explica los temas?
		\item ¿Cómo calificarías la atención del instructor hacia los alumnos para resolver dudas?
		\item ¿En qué puede mejorar el instructor?
		\item Escribe un comentario para el instructor
	\end{enumerate}
	
	\subsection{Encuesta al curso}
	Se proponen 10 preguntas para evaluar el contenido del curso.
	\begin{enumerate}
		\item ¿Cómo te enteraste del curso?
		\item Volverías a tomar un curso de Blockchain como el que recibiste
		\item ¿Qué te pareció el contenido del curso?
		\item ¿Qué te pareció el material del curso?
		\item ¿Los ejemplos que se vieron a lo largo del curso fueron claros?
		\item ¿Qué te pareció la distribución del contenido?
		\item Considerarías que se cumplió el objetivo del curso
		\item ¿Se tocaron todos los temas del programa?
		\item ¿El examen coincide con lo visto en el curso?
		\item Escribe un comentario general de tu experiencia en el curso
	\end{enumerate}
	
	\subsection{Encuesta a la organización}
	\begin{enumerate}
		\item ¿Qué podemos hacer para mejorar nuestros cursos?
		\item Qué actividades propones para dar a conocer y capacitar a nuestros estudiantes en Blockchain
		\item Escribe un comentario de tu experiencia en PROTECO-BAM
	\end{enumerate}

\section{Preguntas frecuentes}
	\subsection{Preguntas de los interesados al adquirir este curso}
	\begin{itemize}
		\item ¿Necesito tener conocimientos previos?\\Se dará un repaso de qué es blockchain sin embargo es preferente haber tomado el curso \textit{Blockchain101}
		\item ¿Es necesario saber programar?\\Es deseable tener al menos los conocimientos básicos de programación 
		\item ¿Este curso tiene que ver con Bitcoin?\\Blockchain es la tecnología que hace funcionar bitcoin pero hay mucho más allá que dicha criptomoneda, hablaremos del proyecto Stellar el cuál está enfocado a realizar transacciones internacionales.
		\item ¿Cuánto cuesta el curso?\\Por el momento este curso es impartido en la UNAM de manera gratuita.
		\item ¿Qué obtengo al finalizar el curso?\\Si acreditas el curso con calificación mayor o igual a 8.0 se te entrega una constancia de asistencia al curso, avalada por la Facultad de Ingeniería, UNAM.
	\end{itemize}
	\subsection{Preguntas de los alumnos a lo largo del curso}

\section{Material general de apoyo}
	\subsection{Artículos}
	\begin{itemize}
		\item \href{https://bitcoin.org/files/bitcoin-paper/bitcoin\_es\_latam.pdf}{Paper de Satoshi Nakamoto}
		\item \href{https://medium.com/@marvin.soto/blockchain-público-vs-blockchain-privado-cuál-es-la-diferencia-8115be4a593b}{Blockchain Pública y Privada}
		
	\end{itemize}
	\subsection{Videos}
	\begin{itemize}
		\item \href{https://www.youtube.com/watch?v=RSj9QQMPMWg}{Privacidad}
		\item \href{https://www.youtube.com/watch?v=E2QT9RHmNsY}{¿Qué es Blockchain?}	
		\item \href{https://www.youtube.com/watch?v=Yn8WGaO\_\_ak\&list=PLgo3Qtdm2bOMKyzFGy-B3MiDAGPIeqbKP\&index=2}{¿Qué es Blockchain en 5 minutos.}
		\item \href{https://www.youtube.com/watch?v=b5dhq3dSG2k}{Las matemáticas de Blockchain}
	\end{itemize}
	\subsection{Sitios web}
	\begin{itemize}
		\item \href{https://coinmarketcap.com/}{CoinMarketCap}
		\item \href{https://www.blockchain.com/charts/n-transactions?timespan=all}{Número de transacciones blockchain}
		\item \href{https://bitso.com/}{Bitso}
		\item \href{https://www.stellar.org/}{Stellar}
	\end{itemize}
	\subsection{Noticias blockchain}
	\begin{itemize}
		\item \href{https://es.cointelegraph.com/}{Cointelegraph}
		\item \href{https://www.criptonoticias.com/}{CriptoNoticias}
	\end{itemize}
	\subsection{Libros}
	\begin{itemize}
		\item \href{https://drive.google.com/file/d/1l-xFaEFFofdnR2wLG76BMCQshgO-MUmB/view?usp=sharing}{Cryptoassets}
		\item \href{https://drive.google.com/file/d/10s1wkh4-nIWxL4mpUM\_vag\_tLxgiellL/view?usp=sharing}{Digital Gold}
		\item \href{https://drive.google.com/file/d/1lXzrKNbJP3V31rd2xHozct1D7vhxqRSN/view?usp=sharing}{Mastering Bitcoin}
		\item \href{https://drive.google.com/file/d/1CjxOQhFaoLQbnxOJEg3o\_DVQcRmmWld3/view?usp=sharing}{Blockchain Revolution}
		\item \href{https://drive.google.com/file/d/1aCVkcd6CdhcTYdfXgkaPOyo2LKv8GeTF/view?usp=sharing}{Blockchain}
	\end{itemize}
\end{document}