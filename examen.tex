\documentclass[a4paper,12pt]{/home/armando/Documentos/Cursos/LaTeX/Plantillas/lib/pub}
%\graphicspath{ {image/} }
\usepackage{wrapfig}
\usepackage[T1]{fontenc}
\usepackage{tgadventor}

\spanishdecimal{.}

\ptcSeccion{Blockchain Academy México}
\ptcTitulo{Stellar}
\ptcLogoSeccion{BAM}

\begin{document}
\putLogo


\protecoTitle
\section{Examen}
\subsection{Fundamentos de Blockchain}
\begin{enumerate}
	\item ¿Qué es blockchain?
	\begin{itemize}
		\item Una base de datos centralizada que almacena solo cadenas.
		\item \textbf{Un conjunto de tecnologías que permite la transferencia de valor de un lugar a otro, sin la ayuda de ningún intermediario a partir de un consenso de red}.
		\item Un conjunto de tecnologías que permite guardar el saldo de las cuentas bancarias de los usuarios con la ayuda de intermediarios.
		\item Una base de datos distribuida controlada por unos pocos, que busca guardar como gasta su dinero la gente.
	\end{itemize}
	\item ¿Qué tecnologías conforman una blockchain pública?
	\begin{itemize}
		\item Red centralizada, libro contable, inteligencia artificial, computer vision.
		\item Criptografía, protocolo de consenso,  redes neuronales, base de datos distribuida.
		\item \textbf{Criptografía, red peer to peer, libro contable, protocolo de consenso}
		\item Base de datos distribuida, criptografía, inteligencia artificial, redes neuronales
	\end{itemize}
	\item ¿Cuál de los siguientes  es un proyecto que ocupe una blockchain pública?
	\begin{itemize}
		\item R3.
		\item \textbf{Stellar}.
		\item Hyperledger.
		\item Ninguno de los anteriores.
	\end{itemize}
\end{enumerate}

\subsection{Introducción a Stellar}
\begin{enumerate}
	\item ¿Qué es Stellar?
	\begin{itemize}
		\item Una plataforma para intercambiar criptomonedas de forma gratuita.
		\item \textbf{Una plataforma de blockchain pública que facilita las transferencias de activos y pagos a nivel global}.
		\item Una blockchain privada para bancos.
		\item Una criptomoneda
	\end{itemize}
	\item ¿Cuál es la meta de Stellar?
	\begin{itemize}
		\item Vender criptomonedas baratas.
		\item Conectar a todos los bancos del mundo.
		\item Que los lumens sean la moneda más utilizada del mundo.
		\item \textbf{Mover dinero a través de fronteras de una manera simple, fácil y rápida}.
	\end{itemize}
	\item ¿Qué se graba en el ledger distribuido de Stellar?
	\begin{itemize}
		\item Historial crediticio de los usuarios.
		\item \textbf{Transacciones de dinero como créditos emitidos por las anclas}.
		\item Los bienes raíces de los usuarios.
		\item Las reglas de negocio de los bancos.
	\end{itemize}
	\item ¿Cuál de las siguientes afirmaciones es falsa?
	\begin{itemize}
		\item \textbf{Stellar es un sistema de pagos nacional}.
		\item Stellar no tiene dueño y pertenece a todos.
		\item Stellar es barato.
		\item Stellar es una red de pares.
	\end{itemize}
	\item ¿Cuál proyecto no ocupa Stellar?
	\begin{itemize}
		\item Satoshi Pay.
		\item \textbf{IBM Food trust}.
		\item IBM World Wire.
		\item Ninguno ocupa Stellar.
	\end{itemize}
	\item Las wallets de Stellar, ¿pueden tener más de un activo asociado?
	\begin{itemize}
		\item \textbf{Verdadero}
		\item Falso
	\end{itemize}
	\item ¿Cuál es el saldo mínimo que debe tener una cuenta en Stellar?
	\begin{itemize}
		\item 1 millón de dolares.
		\item \textbf{1 Lumens}.
		\item 1 bitcoin.
		\item 1 peso.
	\end{itemize}
	\item  ¿Stellar es un sistema para rastrear la propiedad?
	\begin{itemize}
		\item \textbf{Verdadero}.
		\item Falso.
	\end{itemize}
	\item ¿Cómo se llaman los tokens nativos de Stellar?
	\begin{itemize}
		\item Bitcoin.
		\item Satoshis.
		\item \textbf{Lumens}.
		\item Stellars coins.
	\end{itemize}
\end{enumerate}
\subsection{Stellar práctico}
\begin{enumerate}
	\item  ¿Que clave necesita el Friendbot de Stellar para poder crear una cuenta?
	\begin{itemize}
		\item \textbf{Clave pública}.
		\item Clave privada.
		\item Clave de banco.
		\item Clave de sol.
	\end{itemize}
	\item  ¿Qué elementos se deben indicar para hacer una transacción en Stellar?
	\begin{itemize}
		\item Llave privada, cantidad, cuenta de origen.
		\item Cuenta de destino, cuenta de origen, llave pública y llave privada
		\item \textbf{Cuenta de destino, asset, cantidad}.
		\item Ninguna de las anteriores.
	\end{itemize}
	\item Principales elementos que almacena una cuenta de Stellar
	\begin{itemize}
		\item \textbf{Saldos de las cuentas y operaciones de esos saldos}.
		\item Nombre y fecha de nacimiento.
		\item Saldo de tu cuenta bancaria y el nombre de tu banco.
		\item Ninguna de las anteriores.
	\end{itemize}
\end{enumerate}
\end{document}